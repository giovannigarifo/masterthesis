% use UTF-8 encoding in editors such as TeXworks
% !TEX encoding = UTF-8 Unicode
% !TEX TS-program = pdflatex

\documentclass[%
    corpo=13.5pt,
    twoside,
%    stile=classica,
    oldstyle,
%    autoretitolo,
    tipotesi=magistrale,
    greek,
    evenboxes
]{toptesi}

\usepackage[utf8]{inputenc}% codifica d'entrata
\usepackage[T1]{fontenc}%    codifica dei font
\usepackage{lmodern}%        scelta dei font

% Vedere la documentazione toptesi-it.pdf per le
% attenzioni che bisogna usare al fine di ottenere un file
% veramente conforme alle norme per l'archiviabilità.

\usepackage{hyperref}
\hypersetup{%
    pdfpagemode={UseOutlines},
    bookmarksopen,
    pdfstartview={FitH},
    colorlinks,
    linkcolor={blue},
    citecolor={blue},
    urlcolor={blue}
  }

%%%%%%% Definizioni locali
\newtheorem{osservazione}{Osservazione}% Standard LaTeX
\ExtendCaptions{english}{Abstract}{Acknowledgements}



\begin{document}\errorcontextlines=9

% set english ad primary language
\english

%%%%%%%%%%%%%%%%%%%%
% BEGIN front page %
%%%%%%%%%%%%%%%%%%%%
\begin{ThesisTitlePage}*

\ateneo{Politecnico di Torino}
\nomeateneo{DEPARTMENT OF CONTROL AND COMPUTER ENGINEERING}
\CorsoDiLaureaIn{Master of Science in}
\corsodilaurea{Computer Engineering}
\TesiDiLaurea{Master Degree Thesis}

\titolo{Deep Learning on Polito Knowledge Graph}
\sottotitolo{Leveraging Relational GCN for link prediction between nodes of a newly built publications graph}

\CandidateName{Candidate}
\candidato{Giovanni \textsc{Garifo}}

\AdvisorName{Supervisors}
\relatore{Prof.~Antonio Vetrò}
\secondorelatore{Prof.~Juan Carlos De Martin}
\sedutadilaurea{\textsc{Academic~Year} 2018-2019}%

\logosede[6cm]{logopolito}
\end{ThesisTitlePage}
%%%%%%%%%%%%%%%%%%
% END front page %
%%%%%%%%%%%%%%%%%%


% offset rilegatura
%\setbindingcorrection{3mm}

\makeatletter
\newenvironment{miadedica}{
    \clearpage
    \if@twoside
        \ifodd\c@page\else\thispagestyle{empty}\null\clearpage\fi
    \fi
    \thispagestyle{empty}%
    \list{}{\labelwidth\z@
    \leftmargin.73\textwidth
    \parindent\z@
    \raggedright\LARGE\itshape}\item[]
    \normalsize
}

\begin{miadedica}
    To Monia\\
    To my Grandfather
\end{miadedica}


\paginavuota
\sommario

Summary here, one page


\ringraziamenti

Acknowledgements here, half page


\tablespagetrue\figurespagetrue % normalmente questa riga non serve ed e' commentata
\indici

\mainmatter

\chapter{Introduction}

\section{Motivation}

Graphs are used to empower some of the most complex IT services available
today. They can be used to represent almost any kind of information, and
they are particurlarly capable of representing the structure of complex
system, thus to express the relations between its elements.
\newline
\newline
In the past ten years, a lot of effort has been put into trying to leverage the power
of graphs to represent human knowledge and to build search tools capable of
query and understand the semantic relations inside such graphs. RDF graphs are a
particular class of graphs that can be used to build knowledge
repositories. Given a domain and an ontology, they allows to build a structured
representaion of the knowledge of such domain.
\newline
\newline
Modern machine learning techniques can be used to mine latent informations
from such graphs. One of the main challenges in this field is how to learn
meaningful representations of the graph nodes that embed the underlying
knowledge. Such representations can be then used to evaluate new
links inside the graph, task commonly known as link prediction, or to classify
unseen nodes. Deep learning techniques have proved to be first class citizens when
dealing with representation learning tasks, being able to learn the latent
representation of nodes without any prior knowledge other than the graph structure,
so as not to require any feature engineering.



\section{Thesis structure}

\subsection{Chapter 2}

\subsection{Chapter 3}

\subsection{Chapter 4}


%\blankpagestyle{headings}



\chapter{Background}
\section{a section}

\subsection{a subsection}

\chapter{State of the art}

\chapter{Approach and Methodology}

\chapter{Development and Implementation}

\chapter{Evaluation}

\chapter{Conclusions}


%%%%%%%%%%%%%%%%%%%%%%
% BEGIN bibliography %
%%%%%%%%%%%%%%%%%%%%%%
\begin{thebibliography}{9}
\bibitem{gal} G.~Galilei, {\em Nuovi studii sugli astri medicei}, Manuzio,
        Venetia, 1612.
\bibitem{tor1} E.~Torricelli, in ``La pressione barometrica'', {\em Strumenti
        Moderni}, Il Porcellino, Firenze, 1606.
\bibitem{tor2} E.~Torricelli e A.~Vasari, in ``Delle misure'', {\em Atti Nuovo
        Cimento}, vol.~III, n.~2 (feb. 1607), p.~27--31.
\bibitem{duane1964} Duane J.T., \emph{Learning Curve Approach To Reliability
		Monitoring}, IEEE Transactions on Aerospace, Vol. 2, pp. 563-566, 1964
\end{thebibliography}


\end{document}