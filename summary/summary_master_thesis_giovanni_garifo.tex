% !TEX encoding = UTF-8 Unicode
% !TEX TS-program = pdflatex

\documentclass[english, 12pt]{article}
\usepackage[utf8]{inputenc}% codifica d'entrata
\usepackage[T1]{fontenc}%    codifica dei font
\usepackage{babel}
\usepackage{lmodern}%        scelta dei font
\usepackage{tabularx}

\newcommand*{\TitleFontSize}{\fontsize{20}{20}}


% Title
\title{\TitleFontSize \vspace{0.0cm}\textbf{Deep Learning on Academic Knowledge Graphs}}

% Supervisors and candidate
\author{
        \begin{tabularx}{\linewidth}{@{}X @{}>{\arraybackslash}r}
                \textbf{Supervisors} & \textbf{Candidate} \\
                Dr.~Antonio Vetrò & Giovanni \textsc{Garifo} \\
                Prof.~Juan Carlos De Martin & \\
        \end{tabularx}
}

% Date
\date{16 December 2019}


%%%%%%%%%%%%%%%%%%%%%%%%%%%%%%%%%%%%%%%%%%%%%%%%%%%%%%%%%%%%%%
\begin{document}

\maketitle


The publication and sharing of research results is one of the main goals of
academic institutions.
In recent years, many efforts have been made to collect and organize the
scientific knowledge through new, more accessible and comprehensive
data repositories.

An example of such tools is the Institutional Research Information
System (IRIS), an institutional publication repository developed by the
Cineca Consortium and used by the Politecnico di Torino to store and share all
the scientific papers published by its researchers.
IRIS allows to explore the papers by searching within a field of study, matching
the search terms with the keywords used by the authors to tag
their publications.

The current implementation has some limitations: being
inserted by the authors, the keywords can be acronyms, can contain
abbreviations, initials written without capital letters, or misspelled words.
In addition, they do not represent unique and field-wide semantic concepts, but
they are simple character strings. As a consequence, the search engine of IRIS
is incapable of correctly retrieve all the publications about a specific
research topic, being unable to match the searched field of study with an
unambiguous semantic entity.
\newline

The goal of this work is to experiment a new approach based on semantic
technologies and deep neural networks in order to address the above-mentioned
limitations and enabling new opportunities to explore insights about the
scientific community of the Politecnico di Torino.

The main contributions are:

\begin{enumerate}
        \item A novel Knowledge Graph (KG) built on top of
        the scholarly data produced by researchers at Politecnico di Torino.
        \item A deep learning algorithm for the prediction of new facts in
        this KG.
        \item A recommendation system for the suggestion of useful insights
        based on such new facts.
\end{enumerate}

Concerning the first contribution, Knowledge Graphs are a particular class of
graphs that are used to semantically describe the human knowledge in a specific
domain by linking semantic entities through labeled and directed edges.
In the latest years many private and public organizations used KGs to organize
and store data in a semantically coherent way. An example is the Google
Knowledge Graph, which is used to enhance the Google search engine and virtual
assistant capabilities, or the Open Academic Graph, a scientific Knowledge
Graph made by Microsoft and AMiner that describes more than three hundred
million academic papers, and is used to study citation networks and
papers content.

To build the academic graph, a dump of the IRIS database has been used as
input data.
The dump contains more than 20,000 papers, each including relevant metadata
like title, authors, abstract, date, type and venue of publication. Each
publication is linked to semantic topics by means of TellMeFirst, a tool
previously developed at the Nexa Center for Internet \& Society.
The semantic topics are added to the Knowledge Graph as uniquely identified
entities, thus solving the problems related to the ambiguity of the keywords
inserted by the publications authors. In addition, the resulting graph links
together publications, researchers, semantic topics and scientific journals.
This new academic graph has been called the Polito Knowledge Graph (PKG).

The availability of the PKG is the enabler for the second contribution, i.e. a
deep learning algorithm capable of taking as input graph data and use it for
predicting non-existent links: this is one of the most challenging tasks in the
field of statistical relational learning for graph data, because obtaining
meaningful predictions is strictly dependent on the ability of the trained
model to embed the graph nodes characteristics.

Relational Graph Convolutional Network (R-GCN) is a recent deep learning
model that proved to be well suited to work with highly irregular structures
such as graphs, and it has been used in this work to empower a link predictor
able to provide predictions about new facts in the knowledge base.
Once such predictions have been obtained, they
were translated into statements and added to the Polito Knowledge Graph.
A manual and sample-based validation confirmed that most of the predicted
facts are meaningful.

Finally, the third contribution is a visualization of the predicted facts
through a recommendation system for the suggestion of useful insights such
as topics matching with researchers interests or journals accepting publications
in their research field. The recommendation system is a first step in the
direction of offering new tools to the researchers for exploring both new
research fields or discovering researchers from other disciplines working
in the same field.

I conclude remarking that the Polito Knowledge Graph and its recommendation
system can be also used by administration offices of the Polito di Torino and
from external parties -- such as private organizations or government agencies --
who may be interested in finding expert profiles in a given research
area, breaking the current silos of keyword-based searches or
discipline-specific knowledge bases.

\end{document}