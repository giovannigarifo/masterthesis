% use UTF-8 encoding in editors such as TeXworks
% !TEX encoding = UTF-8 Unicode
% !TEX TS-program = pdflatex

\documentclass[%
    corpo=13.5pt,
    twoside,
%    stile=classica,
    oldstyle,
%    autoretitolo,
    tipotesi=magistrale,
    greek,
    evenboxes
]{toptesi}

\usepackage[utf8]{inputenc}% codifica d'entrata
\usepackage[T1]{fontenc}%    codifica dei font
\usepackage{lmodern}%        scelta dei font
\usepackage{listings}       % code listing
\usepackage{mathtools}      % math
\usepackage{eucal}          % math calligraphy
\usepackage{amsfonts}       % blackboard bold letters (like R for real values)
\usepackage{url}            % URLs usage: \url{https://example.com}
\usepackage{cite}           % cite bibtex entries
\usepackage{hyperref}       % internal references to text, chapters ecc

\usepackage{tabularx, booktabs}       % tables
\newcolumntype{b}{X}
\newcolumntype{m}{>{\hsize=.5\hsize}X}
\newcolumntype{s}{>{\hsize=.4\hsize}X}

\usepackage{color}
\definecolor{gray}{rgb}{0.4,0.4,0.4}
\definecolor{darkblue}{rgb}{0.0,0.0,0.6}
\definecolor{cyan}{rgb}{0.0,0.6,0.6}
\definecolor{codebackground}{rgb}{0.95,0.95,0.92}


\lstset{
  basicstyle=\ttfamily,
  columns=fullflexible,
  showstringspaces=false,
  commentstyle=\color{gray}\upshape
}

\lstdefinelanguage{XML}
{
  morestring=[b]",
  morestring=[s]{>}{<},
  morecomment=[s]{<?}{?>},
  stringstyle=\color{black},
  identifierstyle=\color{darkblue},
  keywordstyle=\color{cyan},
  morekeywords={xmlns,version}
}

% Vedere la documentazione toptesi-it.pdf per le
% attenzioni che bisogna usare al fine di ottenere un file
% veramente conforme alle norme per l'archiviabilità.

\usepackage{hyperref}
\hypersetup{%
    pdfpagemode={UseOutlines},
    bookmarksopen,
    pdfstartview={FitH},
    colorlinks,
    linkcolor={blue},
    citecolor={blue},
    urlcolor={blue}
  }

%%%%%%% Definizioni locali
\newtheorem{osservazione}{Osservazione}% Standard LaTeX
\ExtendCaptions{english}{Abstract}{Acknowledgements}



\begin{document}\errorcontextlines=9

% set english ad primary language
\english

%%%%%%%%%%%%%%%%%%%%
% BEGIN front page %
%%%%%%%%%%%%%%%%%%%%
\begin{ThesisTitlePage}*

\ateneo{Politecnico di Torino}
\nomeateneo{DEPARTMENT OF CONTROL AND COMPUTER ENGINEERING}
\CorsoDiLaureaIn{Master of Science in}
\corsodilaurea{Computer Engineering}
\TesiDiLaurea{Master Degree Thesis}

\titolo{Deep Learning on Academic Knowledge Graphs}
\sottotitolo{
    Predicting new facts in a novel semantic graph built
    on top of the Politecnico di Torino scholarly data
}

\CandidateName{Candidate}
\candidato{Giovanni \textsc{Garifo}}

\AdvisorName{Supervisors}
\relatore{Prof.~Antonio Vetrò}
\secondorelatore{Prof.~Juan Carlos De Martin}
\sedutadilaurea{\textsc{Academic~Year} 2018-2019}%

\logosede[6cm]{logopolito}
\end{ThesisTitlePage}
%%%%%%%%%%%%%%%%%%
% END front page %
%%%%%%%%%%%%%%%%%%


% offset rilegatura
%\setbindingcorrection{3mm}

\makeatletter
\newenvironment{miadedica}{
    \clearpage
    \if@twoside
        \ifodd\c@page\else\thispagestyle{empty}\null\clearpage\fi
    \fi
    \thispagestyle{empty}%
    \list{}{\labelwidth\z@
    \leftmargin.73\textwidth
    \parindent\z@
    \raggedright\LARGE\itshape}\item[]
    \normalsize
}

\sommario

To be written.




\end{document}
