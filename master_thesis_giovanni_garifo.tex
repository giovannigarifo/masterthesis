% use UTF-8 encoding in editors such as TeXworks
% !TEX encoding = UTF-8 Unicode
% !TEX TS-program = pdflatex

\documentclass[%
    corpo=13.5pt,
    twoside,
%    stile=classica,
    oldstyle,
%    autoretitolo,
    tipotesi=magistrale,
    greek,
    evenboxes
]{toptesi}

\usepackage[utf8]{inputenc}% codifica d'entrata
\usepackage[T1]{fontenc}%    codifica dei font
\usepackage{lmodern}%        scelta dei font

% Vedere la documentazione toptesi-it.pdf per le
% attenzioni che bisogna usare al fine di ottenere un file
% veramente conforme alle norme per l'archiviabilità.

\usepackage{hyperref}
\hypersetup{%
    pdfpagemode={UseOutlines},
    bookmarksopen,
    pdfstartview={FitH},
    colorlinks,
    linkcolor={blue},
    citecolor={blue},
    urlcolor={blue}
  }

%%%%%%% Definizioni locali
\newtheorem{osservazione}{Osservazione}% Standard LaTeX
\ExtendCaptions{english}{Abstract}{Acknowledgements}



\begin{document}\errorcontextlines=9

% set english ad primary language
\english

%%%%%%%%%%%%%%%%%%%%
% BEGIN front page %
%%%%%%%%%%%%%%%%%%%%
\begin{ThesisTitlePage}*

\ateneo{Politecnico di Torino}
\nomeateneo{DEPARTMENT OF CONTROL AND COMPUTER ENGINEERING}
\CorsoDiLaureaIn{Master of Science in}
\corsodilaurea{Computer Engineering}
\TesiDiLaurea{Master Degree Thesis}

\titolo{Deep Learning on Polito Knowledge Graph}
\sottotitolo{Leveraging Relational GCN for link prediction between nodes of a newly built publications graph}

\CandidateName{Candidate}
\candidato{Giovanni \textsc{Garifo}}

\AdvisorName{Supervisors}
\relatore{Prof.~Antonio Vetrò}
\secondorelatore{Prof.~Juan Carlos De Martin}
\sedutadilaurea{\textsc{Academic~Year} 2018-2019}%

\logosede[6cm]{logopolito}
\end{ThesisTitlePage}
%%%%%%%%%%%%%%%%%%
% END front page %
%%%%%%%%%%%%%%%%%%


% offset rilegatura
%\setbindingcorrection{3mm}

\makeatletter
\newenvironment{miadedica}{
    \clearpage
    \if@twoside
        \ifodd\c@page\else\thispagestyle{empty}\null\clearpage\fi
    \fi
    \thispagestyle{empty}%
    \list{}{\labelwidth\z@
    \leftmargin.73\textwidth
    \parindent\z@
    \raggedright\LARGE\itshape}\item[]
    \normalsize
}

\begin{miadedica}
    To Monia\\
    To my Grandfather
\end{miadedica}


\paginavuota
\sommario

Summary here, one page


\ringraziamenti

Acknowledgements here, half page


\tablespagetrue\figurespagetrue % normalmente questa riga non serve ed e' commentata
\indici

\mainmatter

\chapter{Introduction}

\section{First section}

first section here

Le grandezze in gioco sono evidenziate nella figura \ref{fig:orbita}.
\begin{figure}[ht]\centering
\setlength{\unitlength}{0.01\textwidth}
\begin{picture}(40,30)(30,0)
\put(50,15){\circle{20}}
\put(47,15){\circle*{1}}
\put(30,0){\line(0,1){30}}
\put(30,30){\line(1,0){40}}
\put(70,30){\line(0,-1){30}}
\put(70,0){\line(-1,0){40}}
\end{picture}
\caption{Orbita del generico satellite; si noti l'eccentricit\`a dell'orbita rispetto al pianeta.}\label{fig:orbita}
\end{figure}

\section{Second section}


%\blankpagestyle{headings}



\chapter{Background}
\section{a section}

\subsection{a subsection}

\chapter{State of the art}

\chapter{Approach and Methodology}

\chapter{Development and Implementation}

\chapter{Evaluation}

\chapter{Conclusions}


%%%%%%%%%%%%%%%%%%%%%%
% BEGIN bibliography %
%%%%%%%%%%%%%%%%%%%%%%
\begin{thebibliography}{9}
\bibitem{gal} G.~Galilei, {\em Nuovi studii sugli astri medicei}, Manuzio,
        Venetia, 1612.
\bibitem{tor1} E.~Torricelli, in ``La pressione barometrica'', {\em Strumenti
        Moderni}, Il Porcellino, Firenze, 1606.
\bibitem{tor2} E.~Torricelli e A.~Vasari, in ``Delle misure'', {\em Atti Nuovo
        Cimento}, vol.~III, n.~2 (feb. 1607), p.~27--31.
\bibitem{duane1964} Duane J.T., \emph{Learning Curve Approach To Reliability
		Monitoring}, IEEE Transactions on Aerospace, Vol. 2, pp. 563-566, 1964
\end{thebibliography}


\end{document}