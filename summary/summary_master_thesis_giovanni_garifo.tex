% use UTF-8 encoding in editors such as TeXworks
% !TEX encoding = UTF-8 Unicode
% !TEX TS-program = pdflatex

\documentclass[%
    corpo=13.5pt,
    twoside,
%    stile=classica,
    oldstyle,
%    autoretitolo,
    tipotesi=magistrale,
    greek,
    evenboxes
]{toptesi}

\usepackage[utf8]{inputenc}% codifica d'entrata
\usepackage[T1]{fontenc}%    codifica dei font
\usepackage{lmodern}%        scelta dei font
\usepackage{listings}       % code listing
\usepackage{mathtools}      % math
\usepackage{eucal}          % math calligraphy
\usepackage{amsfonts}       % blackboard bold letters (like R for real values)
\usepackage{url}            % URLs usage: \url{https://example.com}
\usepackage{cite}           % cite bibtex entries
\usepackage{hyperref}       % internal references to text, chapters ecc

\usepackage{tabularx, booktabs}       % tables
\newcolumntype{b}{X}
\newcolumntype{m}{>{\hsize=.5\hsize}X}
\newcolumntype{s}{>{\hsize=.4\hsize}X}

\usepackage{color}
\definecolor{gray}{rgb}{0.4,0.4,0.4}
\definecolor{darkblue}{rgb}{0.0,0.0,0.6}
\definecolor{cyan}{rgb}{0.0,0.6,0.6}
\definecolor{codebackground}{rgb}{0.95,0.95,0.92}


\lstset{
  basicstyle=\ttfamily,
  columns=fullflexible,
  showstringspaces=false,
  commentstyle=\color{gray}\upshape
}

\lstdefinelanguage{XML}
{
  morestring=[b]",
  morestring=[s]{>}{<},
  morecomment=[s]{<?}{?>},
  stringstyle=\color{black},
  identifierstyle=\color{darkblue},
  keywordstyle=\color{cyan},
  morekeywords={xmlns,version}
}

% Vedere la documentazione toptesi-it.pdf per le
% attenzioni che bisogna usare al fine di ottenere un file
% veramente conforme alle norme per l'archiviabilità.

\usepackage{hyperref}
\hypersetup{%
    pdfpagemode={UseOutlines},
    bookmarksopen,
    pdfstartview={FitH},
    colorlinks,
    linkcolor={blue},
    citecolor={blue},
    urlcolor={blue}
  }

%%%%%%% Definizioni locali
\newtheorem{osservazione}{Osservazione}% Standard LaTeX
\ExtendCaptions{english}{Abstract}{Acknowledgements}



\begin{document}\errorcontextlines=9

% set english ad primary language
\english

%%%%%%%%%%%%%%%%%%%%
% BEGIN front page %
%%%%%%%%%%%%%%%%%%%%
\begin{ThesisTitlePage}*

\ateneo{Politecnico di Torino}
\nomeateneo{DEPARTMENT OF CONTROL AND COMPUTER ENGINEERING}
\CorsoDiLaureaIn{Master of Science in}
\corsodilaurea{Computer Engineering}
\TesiDiLaurea{Master Degree Thesis}

\titolo{Deep Learning on Academic Knowledge Graphs}
\sottotitolo{
    Predicting new facts in a novel semantic graph built
    on top of the Politecnico di Torino scholarly data
}

\CandidateName{Candidate}
\candidato{Giovanni \textsc{Garifo}}

\AdvisorName{Supervisors}
\relatore{Prof.~Antonio Vetrò}
\secondorelatore{Prof.~Juan Carlos De Martin}
\sedutadilaurea{\textsc{Academic~Year} 2018-2019}%

\logosede[6cm]{logopolito}
\end{ThesisTitlePage}
%%%%%%%%%%%%%%%%%%
% END front page %
%%%%%%%%%%%%%%%%%%


% offset rilegatura
%\setbindingcorrection{3mm}

\makeatletter
\newenvironment{miadedica}{
    \clearpage
    \if@twoside
        \ifodd\c@page\else\thispagestyle{empty}\null\clearpage\fi
    \fi
    \thispagestyle{empty}%
    \list{}{\labelwidth\z@
    \leftmargin.73\textwidth
    \parindent\z@
    \raggedright\LARGE\itshape}\item[]
    \normalsize
}

\sommario


The publication and sharing of new research results is one of the
main goal of an academic institution. In recent years, many efforts have been
made to collect and organize the scientific knowledge through new,
comprehensive data repositories.
To achieve such goal new tools that are able not only to store data, but also to 
describe them are needed.

Knowledge graphs are a particular class of graphs that are used to
semantically describe the human knowledge in a specific domain by linking
semantic entities through labeled and directed edges.
In the latest years many private and public organizations have used such 
new kind of data structure to organize and store semantic information.
An example is the \emph{Google Knowledge Graph}, which is used to enhance the 
Google search engine and virtual assistant capabilities, allowing to 
retrieve punctual information about everything that has been classified 
in its ontology and described in its knowledge base.
Another example is the \emph{Open Academic Graph}, a Scientific Knowledge Graph 
made by Microsoft and AMiner that collects more then three hundred 
million academic papers, which is used to study citation networks and 
papers content.

In our work we present a novel semantic graph built on top of the
scholarly data produced by the Politecnico di Torino researchers, and how we
employed state-of-the-art machine learning techniques for the prediction
of new facts in this new knowledge base.
Such graph, built by leveraging Semantic Web and Natural Language Processing 
technologies, links together publications, researchers, fields of study 
and scientific journals in order to build a knowledge base that describes 
the Politecnico di Torino scientific community.
We decided to call such new academic graph the \emph{Polito Knowledge Graph}.

We built such graph starting from the metadata made available by the 
IRIS web portal, which collects all the scientific papers published by 
the Politecnico di Torino researchers.
We wanted to link each publication to its research topics, to do so 
we employed TellMeFirst, a tool for the automatic extraction of semantic 
concepts from a text, which uses DBpedia as its source of knowledge.
Using the publications abstracts as input text for TellMeFirst, we
have been able to extract such semantic topic and add them as 
entities in the knowledge graph.
\newline

The availability of such a complex and informative data structure leads
to the opening of interesting scenarios, especially when thinking about
the latent information that can be extracted from it.

In recent years, efforts have been made to develop machine learning 
algorithms capable of taking as input graph data, both for the classification 
of unseen nodes and for the prediction of new links.

The prediction of non-existent links between graph nodes is one of the
most challenging tasks in the field of statistical relational learning for graph 
data, mainly because, in order to obtain meaningful predictions, the vector 
representations of the graph nodes must embed their semantic characteristics.





To accomplish such goal, we decided to employ Deep Learning architectures derived
from the image recognition field and specifically adapted to the task of
representation learning for graph data.
Such architectures allowed us to obtain representations that have been
directly learnt from the graph structure itself, without requiring any prior
knowledge or feature engineering.

Using such learnt representations, we have been able to obtain meaningful
predictions about new, unseen facts in the knowledge base.
We used such predictions to complete the information present in the
knowledge graph and to build a recommendation system for the suggestion of useful
insights to the Politecnico di Torino researchers.



\end{document}